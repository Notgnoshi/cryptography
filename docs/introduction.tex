% !TEX root = docs.tex

\subsection{About the \texttt{crypto} Library}
    This is Austin Gill's final project for CSC 512, Cryptography from the South Dakota School of Mines and Technology taught by Dr.\ Christer Karlsson. The project is to be a portfolio showcasing significant portions of the textbook.

    The repository is laid out as follows

    \begin{figure}[H]
        \dirtree{%
        .1 project.
            .2 crypto/\DTcomment{Top level library source code}.
                .3 classical/\DTcomment{\texttt{classical} ciphers submodule}.
                .3 \ldots.
            .2 docs/\DTcomment{Library documentation}.
            .2 homework/\DTcomment{Course homework assignments and code snippets}.
            .2 tests/\DTcomment{Library unit tests}.
                .3 \ldots.
        }
        \caption{Repository Layout}\label{repo-layout}
    \end{figure}

    Example usage, provided \texttt{crypto/} is in your system path.

    \begin{minted}{python}
        from crypto.classical import AffineCipher
        from crypto.random import generate_alpha

        random_plaintext = generate_alpha(100)
        affine = AffineCipher(a=9, b=18)
        print(affine.encrypt(random_plaintext))
    \end{minted}

\subsection{About the Documentation}
    This documentation is intended to supplement the \texttt{crypto} library by adding usage documentation and examples, as well as provide in depth implementation explanations.

    This documentation may be built by running \texttt{make} in the \texttt{docs/} directory.

\subsection{Dependecies}
    \subsubsection{Documentation}
        Building this documentation has dependencies on
        \begin{itemize}
            \item \texttt{make}
            \item \texttt{latexmk}
            \item \texttt{minted}
            \item \texttt{python-pygments}
        \end{itemize}

    \subsubsection{\texttt{crypto} Library}
        The \texttt{crypto} Python Library has dependencies on
        \begin{itemize}
            \item Python 3.5+
            \item \texttt{gmpy2} 2.0.8+
            \item \texttt{numpy} 1.13.1+
        \end{itemize}

\subsection{Unit Tests}
    The \texttt{crypto} library has a unit test suite runnable {\em from the root repository directory\/} by \texttt{python3 tests/runtests.py}. Otherwise the \texttt{crypto} library will not be in your path.

    The \texttt{tests/} directory contains the \texttt{runtests.py} script and one test file per submodule.

    For example, the \texttt{tests/test\textunderscore{}classical\textunderscore{}ciphers.py} file contains the following test (among others) taken from the textbook.

    \begin{minted}{python}
        class AffineCipherTest(unittest.TestCase):
            def test_affine_encrypt(self):
                plaintext = 'affine'
                affine = AffineCipher(9, 2)
                ciphertext = affine.encrypt(plaintext)
                self.assertEqual(ciphertext, 'cvvwpm')
    \end{minted}
