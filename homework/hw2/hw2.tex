\documentclass[12pt]{article}
\usepackage{amsmath}
\usepackage{amssymb}
\usepackage{amsthm}
\usepackage{cancel}
\usepackage{color}
% Don't use the dumb paragraph indentation. For repeated breaks in mathematical chains the parskips
% suck.
\usepackage{parskip}
\usepackage{minted}
\usepackage[margin=1in]{geometry}
\usepackage[hidelinks]{hyperref}

\setminted{autogobble=true, style=tango, breaklines}

\renewcommand{\thesubsection}{\thesection.\alph{subsection}}
\renewcommand{\mod}[1]{\mathrm{mod}\ #1}
\renewcommand{\pmod}[1]{\ (\mod{#1})}
\renewcommand{\qedsymbol}{$\blacksquare$}

\newcommand{\Z}{\mathbb{Z}}
\newtheorem*{defn}{Definition}
\newtheorem*{thm}{Theorem}


\title{Homework 2}
\author{Austin Gill}

\begin{document}
\maketitle

\section{} \textit{Let $p$ be prime. Show that $a^p \equiv a \pmod{p}$ for all $a$.}

    \begin{proof}
        Let $p$ be prime, and further suppose $p$ does not divide $a$. Then from Fermat's Little Theorem, we have

        \begin{align*}
            a^{p - 1} &\equiv 1 &\pmod{p}\\
            a \cdot a^{p - 1} &\equiv a \cdot 1 &\pmod{p}\\
            a^p &\equiv a &\pmod{p}\\
        \end{align*}

        Thus, as long as $p \nmid a$, we have $a^p \equiv a \pmod{p}$.
    \end{proof}

\section{} \textit{Let $p \geq 3$ be prime. Show that the only solutions to $x^2 \equiv 1 \pmod{p}$ are $x \equiv \pm 1 \pmod{p}$.}

    \begin{proof}
        Let $p$ be an odd prime. Then

        \begin{align*}
            x^2 &\equiv 1 &\pmod{p}\\
            x^2 - 1 &\equiv 0 &\pmod{p}\\
            (x + 1)(x - 1) &\equiv 0 &\pmod{p}\\
        \end{align*}

        Since $p$ is prime, $x$ must be $\pm 1$.
    \end{proof}

\section{}
    \subsection{} \textit{Find all four solutions to $x^2 \equiv 133 \pmod{143}$.}
    \subsection{} \textit{Find all two solutions to $x^2 \equiv 77 \pmod{143}$.}

        The Python snippet below solves both equations

        \inputminted{python}{hw2_3.py}

        and outputs the following:

        \begin{minted}{python}
            [43, 56, 87, 100]
            [44, 99]
        \end{minted}

\section{}
    \subsection{} \textit{Find solutions to $3x - 15y = 2$.}

        Note $\gcd(3, 15) = 3$, which does not divide 2. Thus there is no integer solution for $3x - 15y = 2$.
    \subsection{} \textit{Find solutions to $3x - 14y = 2$.}

        However, $\gcd(3, -14) = -1$. The following

        \begin{minted}{python}
            from crypto.math import extended_gcd
            g, x, y = extended_gcd(3, -14)
        \end{minted}

        gives $x = -5$ and $y = -1$. (note 14 is positive) Thus we have $$3 \cdot (-5) - 14 \cdot (-1) = -1.$$ Multiplying both sides by $-2$, we get $$2 \cdot 3 \cdot 5 - 2 \cdot 14 \cdot 1 = 2,$$ or rather $$3 \cdot 10 - 14 \cdot 2 = 2$$

        Thus, $x = 10$ and $y = 2$ is a solution.

\section{} \textit{Prove the following theorem.}
    \begin{thm}
        Let $p$ be a positive prime and $g$ be a primitive root modulo $p$.
        \begin{enumerate}
            \item Let $n$ be an integer, then $$g^n \equiv 1 \pmod{p} \Longleftrightarrow n \equiv 0 \pmod{p - 1}$$
            \item Let $j$ and $k$ be integers, then $$g^j \equiv g^k \pmod{p} \Longleftrightarrow j \equiv k \pmod{p - 1}$$
        \end{enumerate}
    \end{thm}

    \begin{proof} Let $p$ be a positive prime and $g$ be a primitive root modulo $p$.
        \begin{enumerate}
            \item Let $n$ be an integer.
            \begin{itemize}
                \item[$(\Longrightarrow)$] Suppose $g^n \equiv 1 \pmod{p}$. We wish to show that $n$ divides $p - 1$. Thus we have
                $$n = (p - 1)q + r$$
                for some remainder $r$ between $0$ and $p - 1$. We wish to show that $r = 0$. We then have
                $$1 \equiv g^n \equiv g^{(p - 1)q + r} \equiv g^{(p - 1)q}g^r \pmod{p}$$
                but by Fermat's Little Theorem, we have
                $$g^{(p - 1)q} \equiv (g^q)^{p - 1} \equiv 1 \pmod{p}$$
                so then we have
                $$1 \equiv g^n \equiv 1 \cdot g^r \equiv g^r \pmod{p}$$

                \item[$(\Longleftarrow)$] Suppose $n \equiv 0 \pmod{p - 1}$.
            \end{itemize}
        \end{enumerate}
    \end{proof}

\section{}
    \subsection{} \textit{Let $p$ be a positive prime. Define a \textbf{primitive root modulo $p$}.}

        \begin{defn}
            A \textbf{primitive root} or \textbf{generator} mod $p$ is a number whose powers mod $p$ generate every element of $\Z_p$.

            Equivalently, suppose $\phi : \Z_p \to \Z_p$ defined by $\phi(x) = a^x \pmod{p}$. $a$ is a primitive root mod $p$ iff $\phi(\Z_p) = \Z_p$.
        \end{defn}

    \subsection{} \textit{Identify all primitive root modulo $11$. Is your solution consistent with the claim that there are $\phi(\phi(p))$ primitive roots modulo $p$?}

        The following
        \inputminted{python}{hw2_6.py}
        gives
        \begin{minted}{python}
            [2, 6, 7, 8]
        \end{minted}
        as the primitive roots of $11$.

        $$\phi(\phi(11)) = \phi(10) = \phi(2 \cdot 5) = (2 - 1)(5 - 1) = 4$$

        Thus my solution is consistent with the above claim.

    \subsection{} \textit{We stated the \textbf{Primitive Root Theorem}:}

        \begin{thm}
            If $p$ is prime, then there is at least one primitive root modulo $p$.
        \end{thm}

        \textit{Show that this result does not hold for a composite number $n$:}

        \begin{thm}
            If $n$ is composite, then there may not be a unit that is a multiplicative generator (primitive root) of the set of units modulo $n$.
        \end{thm}

        \textit{Hint: Check modulo $8$.}

\section{} \textit{Question 3.27 in the book.}
\end{document}
