\documentclass[12pt]{article}
\usepackage{amsmath}
\usepackage{amssymb}
\usepackage{amsthm}
\usepackage{cancel}
\usepackage{color}
% Don't use the dumb paragraph indentation. For repeated breaks in mathematical chains the parskips
% suck.
\usepackage{parskip}
\usepackage{minted}
\usepackage[margin=1in]{geometry}
\usepackage[hidelinks]{hyperref}

\setminted{autogobble=true, style=tango, breaklines}

\renewcommand{\thesubsection}{\thesection.\alph{subsection}}
\renewcommand{\mod}[1]{\mathrm{mod}\ #1}
\renewcommand{\pmod}[1]{\ (\mod{#1})}
\renewcommand{\qedsymbol}{$\blacksquare$}

\newcommand{\Z}{\mathbb{Z}}
\newtheorem*{defn}{Definition}
\newtheorem*{thm}{Theorem}


\title{Homework 2}
\author{Austin Gill}

\begin{document}
\maketitle

\section{} \textit{Let $p$ be prime. Show that $a^p \equiv a \pmod{p}$ for all $a$.}

    \begin{proof}
        Let $p$ be prime, and further suppose $p$ does not divide $a$. Then from Fermat's Little Theorem, we have

        \begin{align*}
            a^{p - 1} &\equiv 1 &\pmod{p}\\
            a \cdot a^{p - 1} &\equiv a \cdot 1 &\pmod{p}\\
            a^p &\equiv a &\pmod{p}\\
        \end{align*}

        Thus, as long as $p \nmid a$, we have $a^p \equiv a \pmod{p}$.
    \end{proof}

\section{} \textit{Let $p \geq 3$ be prime. Show that the only solutions to $x^2 \equiv 1 \pmod{p}$ are $x \equiv \pm 1 \pmod{p}$.}

    \begin{proof}
        Let $p$ be an odd prime. Then

        \begin{align*}
            x^2 &\equiv 1 &\pmod{p}\\
            x^2 - 1 &\equiv 0 &\pmod{p}\\
            (x + 1)(x - 1) &\equiv 0 &\pmod{p}\\
        \end{align*}

        Since $p$ is prime, $x$ must be $\pm 1$.
    \end{proof}

\section{}
    \subsection{} \textit{Find all four solutions to $x^2 \equiv 133 \pmod{143}$.}
    \subsection{} \textit{Find all two solutions to $x^2 \equiv 77 \pmod{143}$.}

        The Python snippet below solves both equations

        \inputminted{python}{hw2_3.py}

        and outputs the following:

        \begin{minted}{python}
            [43, 56, 87, 100]
            [44, 99]
        \end{minted}

\section{}
    \subsection{} \textit{Find solutions to $3x - 15y = 2$.}

        Note $\gcd(3, 15) = 3$, which does not divide 2. Thus there is no integer solution for $3x - 15y = 2$.
    \subsection{} \textit{Find solutions to $3x - 14y = 2$.}

        However, $\gcd(3, -14) = -1$. The following

        \begin{minted}{python}
            from crypto.math import extended_gcd
            g, x, y = extended_gcd(3, -14)
        \end{minted}

        gives $x = -5$ and $y = -1$. (note 14 is positive) Thus we have $$3 \cdot (-5) - 14 \cdot (-1) = -1.$$ Multiplying both sides by $-2$, we get $$2 \cdot 3 \cdot 5 - 2 \cdot 14 \cdot 1 = 2,$$ or rather $$3 \cdot 10 - 14 \cdot 2 = 2$$

        Thus, $x = 10$ and $y = 2$ is a solution.

\section{} \textit{Prove the following theorem.}
    \begin{thm}
        Let $p$ be a positive prime and $g$ be a primitive root modulo $p$.
        \begin{enumerate}
            \item Let $n$ be an integer, then $$g^n \equiv 1 \pmod{p} \Longleftrightarrow n \equiv 0 \pmod{p - 1}$$

            \item Let $j$ and $k$ be integers, then $$g^j \equiv g^k \pmod{p} \Longleftrightarrow j \equiv k \pmod{p - 1}$$
        \end{enumerate}
    \end{thm}

    \begin{proof} Let $p$ be a positive prime and $g$ be a primitive root modulo $p$.
        \begin{enumerate}
            \item Let $n$ be an integer.
            \begin{itemize}
                \item[$(\Longrightarrow)$] Suppose $g^n \equiv 1 \pmod{p}$. We wish to show that $n$ divides $p - 1$. Thus we have
                $$n = (p - 1)q + r$$
                for some remainder $r$ between $0$ and $p - 1$. We wish to show that $r = 0$. We then have
                $$1 \equiv g^n \equiv g^{(p - 1)q + r} \equiv g^{(p - 1)q}g^r \pmod{p}$$
                but by Fermat's Little Theorem, we have
                $$g^{(p - 1)q} \equiv (g^q)^{p - 1} \equiv 1 \pmod{p}$$
                so then we have
                $$1 \equiv g^n \equiv 1 \cdot g^r \equiv g^r \pmod{p}$$
                We wish to show that this implies $r = 0$. So by way of contradiction, suppose $r \neq 0$. But $0 \leq r < p - 1$, so rather, suppose $r > 0$.

                Consider then, the powers of $g \pmod{p}$. $g$ is a primitive root, or a generator, of $p$, so
                \begin{align*}
                    1 &\equiv g^r &\pmod{p}\\
                    g \cdot 1\equiv g &\equiv g \cdot g^r \equiv g^{r + 1} &\pmod{p}\\
                    g^2 &\equiv g^{r + 2} &\pmod{p}\\
                    &\vdots &\vdots
                \end{align*}
                Thus the powers of $g$ generate only the $r$ numbers $g, g^2, \dots, 1$. However, $r$ must be less than $p - 1$, so not every number mod $p$ may be a power of $g$. This is a contradiction. Thus $r$ must equal $0$. Therefore $p - 1$ must divide $n$.

                \item[$(\Longleftarrow)$] Suppose $n \equiv 0 \pmod{p - 1}$. Then $n = (p - 1)q$ for some $q$. Thus we have
                $$g^n \equiv g^{(p - 1)q} \equiv {(g^q)}^{p - 1} \pmod{p}$$
                By Fermat's Little Theorem we have
                $$g^n \equiv {(g^q)}^{p - 1} \equiv 1 \pmod{p}$$
                which is the result we wished for.
            \end{itemize}

            \item Let $j, k \in \Z$, and without loss of generality suppose $j \geq k$.
            \begin{itemize}
                \item[$(\Longrightarrow)$] Suppose $g^j \equiv g^k \pmod{p}$. We wish to show that $j \equiv k \pmod{p - 1}$. We have
                \begin{align*}
                    g^j &\equiv g^k &\pmod{p}\\
                    \frac{g^j}{g^k} &\equiv 1 &\pmod{p}\\
                    g^{j - k} &\equiv 1 &\pmod{p}\\
                \end{align*}
                By part (1), we have $j - k \equiv 0 \pmod{p - 1}$, implying that $j \equiv k \pmod{p - 1}$

                \item[$(\Longleftarrow)$] Suppose $j \equiv k \pmod{p - 1}$. We wish to show that $g^j \equiv g^k \pmod{p}$. If $j \equiv k \pmod{p - 1}$, then $j - k \equiv 0 \pmod{p - 1}$. Again by part (1), we have that $g^{j - k} \equiv 1 \pmod{p}$.
                \begin{align*}
                    g^{j - k} &\equiv 1 &\pmod{p}\\
                    g^k \cdot g^{j - k} &\equiv g^k \cdot 1 &\pmod{p}\\
                    g^{k + j - k} \equiv g^j &\equiv g^k &\pmod{p}\\
                \end{align*}
                Thus we have $g^j \equiv g^k \pmod{p}$.
            \end{itemize}
        \end{enumerate}
    \end{proof}

\section{}
    \subsection{}
        \begin{defn}
            A \textbf{primitive root} or \textbf{generator} mod $p$ is a number whose powers mod $p$ generate every element of $\Z_p$.

            Equivalently, suppose $\phi : \Z_p \setminus \{0\} \to \Z_p \setminus \{0\}$ defined by $\phi(x) = a^x \pmod{p}$. $a$ is a primitive root mod $p$ iff $\phi(\Z_p \setminus \{0\}) = \Z_p \setminus \{0\}$.
        \end{defn}

        This is my understanding from discussion in class and the textbook. However, looking up the definition gives this:

        \begin{defn}
            A number $g$ is a primitive root of $n$ if every number $a$ that is coprime with $n$ is congruents to some power of $g$ mod $n$.

            Equivalently, for every integer $a$ coprime with $n$, there exists a $k \in \Z$ such that $g^k \equiv a \pmod{n}$.
        \end{defn}

        This impacts the theorem stated in part (c), because every number less than a prime $p$ is coprime with $p$. This is not true for composite numbers. It also impacts the implementation of the following Python script.

    \subsection{} \textit{Identify all primitive root modulo $11$. Is your solution consistent with the claim that there are $\phi(\phi(p))$ primitive roots modulo $p$?}

        The following
        \inputminted{python}{hw2_6.py}
        gives
        \begin{minted}{python}
            [2, 6, 7, 8]
        \end{minted}
        as the primitive roots of $11$.

        $$\phi(\phi(11)) = \phi(10) = \phi(2 \cdot 5) = (2 - 1)(5 - 1) = 4$$

        Thus my solution is consistent with the above claim.

        \subsubsection{Alternate Script Implementation}
        \begin{minted}{python}
            def primitive_roots(m):
                """Yields the primitive roots of `m`"""
                for a in range(1, m):
                    if set(powmod(a, p, m) for p in range(1, m)) == set(range(1, m)):
                        yield a
        \end{minted}

        This implementation also yields \texttt{[2, 6, 7, 8]} as the primitive roots of $11$, but does not allow composite numbers to have primitive roots.

    \subsection{} \textit{We stated the \textbf{Primitive Root Theorem}:}

        \begin{thm}
            If $p$ is prime, then there is at least one primitive root modulo $p$.
        \end{thm}

        \textit{Show that this result does not hold for a composite number $n$:}

        \begin{thm}
            If $n$ is composite, then there may not be a unit that is a multiplicative generator (primitive root) of the set of units modulo $n$.
        \end{thm}

        \textit{Hint: Check modulo $8$.}

        The above Python snippet yields an empty list as the primitive roots of $8$. However, it yields the list \texttt{[3, 7]} as the primitive roots of $10$. Thus there exists a unit that is a primitive root of a composite $n$, and there does not exist a unit that is a primitive root of a different composite $m$.

\section{} Alice designs a cryptosystem as follows.
    \begin{itemize}
        \item She chooses two distinct primes $p$ and $q$ congruent to $3 \mod 4$, and keeps them secret.
        \item She makes $n = pq$ public.
        \item When Bob wants to send Alice a message $m$, he computes $x = m^2 \pmod{n}$ and sends $x$ to Alice.
        \item Alice decrypts the message as follows:
        \begin{enumerate}
            \item Given a number $x$, compute the square roots of $x \pmod{n}$. There will usually be more than one square root. Possible because Alice knows $p$ and $q$.
            \item Choose a square root at random. Assume this is $m$.
            \begin{enumerate}
                \item If the message is meaningful, stop.
                \item Otherwise pick a different random square root.
            \end{enumerate}
        \end{enumerate}
    \end{itemize}

    \subsection{} \textit{Why should Alice expect to get a meaningful message fairly soon?}

        There are only four possible solutions to $m^2 \equiv x \pmod{pq}$ obtainable by the Chinese Remainder Theorem:

        \begin{figure}[H]
            \centering
            \begin{tikzpicture}[>=stealth, node distance=4cm]
                \node[] (A) {$m^2 \equiv x \pmod{pq}$};
                \node[below left of=A] (B) {$\sqrt{m^2} \equiv \sqrt{x} \pmod{p}$};
                \node[below right of=A] (C) {$\sqrt{m^2} \equiv \sqrt{x} \pmod{q}$};
                \node[below of=B] (D) {$m \equiv \pm \sqrt{x} \equiv x^{\frac{1}{4}(p + 1} \pmod{p}$};
                \node[below of=C] (E) {$m \equiv \pm \sqrt{x} \equiv x^{\frac{1}{4}(q + 1} \pmod{q}$};

                \draw[->] (A) -- (B);
                \draw[->] (A) -- (C);
                \draw[->] (B) -- (D);
                \draw[->] (C) -- (E);
            \end{tikzpicture}
            \caption{The solutions of $m^2 \equiv x \pmod{pq}$}
        \end{figure}

    \subsection{} \textit{If Oscar intercepts $x$, and already knows $n$, why should it be hard to determine $m$?}

        The primary difficulty of this problem lies in factoring $n$ into $p$ and $q$, something that Alice already knows.

    \subsection{} \textit{If Eve breaks into Alice's office and is able to execute chosen-ciphertext attacks on Alice's decryption machine, how can she determine the factorization of $n = pq$?}

    According to the statement on page 88 of the textbook, finding the four solutions $\pm a$, $\pm b$ to $x^2 \equiv y \pmod{n}$ is computationally equivalent to factoring $n$. Alice's decryption machine effectively returns the four solutions $\pm a$, $\pm b$. We know that $a \equiv b \pmod{p}$, so then $\gcd(a - b, n) = p$. Finding $q$ is a simple matter from there.
\end{document}
