\documentclass[12pt]{article}
\usepackage{amsmath, amssymb, amsthm, cancel, color, parskip, minted}
\usepackage[margin=1in]{geometry}
\usepackage[hidelinks]{hyperref}

\setminted{autogobble=true, style=tango}
\renewcommand{\thesubsection}{\thesection.\alph{subsection}}

\renewcommand{\mod}[1]{\mathrm{mod}\ #1}
\renewcommand{\pmod}[1]{\ (\mod{#1})}
\newtheorem*{defn}{Definition}
\newtheorem*{thm}{Theorem}

\title{Homework 2}
\author{Austin Gill}

\begin{document}
\maketitle

\section{} \textit{Let $p$ be prime. Show that $a^p \equiv a \pmod{p}$ for all $a$.}
\section{} \textit{Let $p \geq 3$ be prime. Show that the only solutions to $x^2 \equiv 1 \pmod{p}$ are $x \equiv \pm 1 \pmod{p}$.}
\section{}
    \subsection{} \textit{Find all four solutions to $x^2 \equiv 133 \pmod{143}$.}
    \subsection{} \textit{Find all two solutions to $x^2 \equiv 77 \pmod{143}$.}
\section{}
    \subsection{} \textit{Find solutions to $3x - 15y = 2$.}
    \subsection{} \textit{Find solutions to $3x - 14y = 2$.}
\section{} \textit{Prove the following theorem.}
    \begin{thm}
        Let $p$ be a position prime and $g$ be a primitive root modulo $p$.
        \begin{enumerate}
            \item Let $n$ be an integer, then $$g^n \equiv 1 \pmod{p} \Longleftrightarrow n \equiv 0 \pmod{p - 1}$$
            \item Let $j$ and $k$ be integers, then $$g^j \equiv g^k \pmod{p} \Longleftrightarrow j \equiv k \pmod{p - 1}$$
        \end{enumerate}
    \end{thm}
\section{}
    \subsection{} \textit{Let $p$ be a positive prime. Define a \textbf{primitive root modulo $p$}.}
    \subsection{} \textit{Identify all primitive root modulo $11$. Is your solution consistent with the claim that there are $\phi(\phi(p))$ primitive roots modulo $p$?}
    \subsection{} \textit{We stated the \textbf{Primitive Root Theorem}:}

        \begin{thm}
            If $p$ is prime, then there is at least one primitive root modulo $p$.
        \end{thm}

        \textit{Show that this result does not hold for a composite number $n$:}

        \begin{thm}
            If $n$ is composite, then there may not be a unit that is a multiplicative generator (primitive root) of the set of units modulo $n$.
        \end{thm}

        \textit{Hint: Check modulo $8$.}

\section{} \textit{Question 3.27 in the book.}
\end{document}
