\documentclass[12pt]{article}
\usepackage{amsmath}
\usepackage{amssymb}
\usepackage{amsthm}
\usepackage{cancel}
\usepackage{color}
% Don't use the dumb paragraph indentation. For repeated breaks in mathematical chains the parskips
% suck.
\usepackage{parskip}
\usepackage{minted}
\usepackage[margin=1in]{geometry}
\usepackage[hidelinks]{hyperref}

\setminted{autogobble=true, style=tango, breaklines}

\renewcommand{\thesubsection}{\thesection.\alph{subsection}}
\renewcommand{\mod}[1]{\mathrm{mod}\ #1}
\renewcommand{\pmod}[1]{\ (\mod{#1})}
\renewcommand{\qedsymbol}{$\blacksquare$}

\newcommand{\Z}{\mathbb{Z}}
\newtheorem*{defn}{Definition}
\newtheorem*{thm}{Theorem}


\usepackage{maplestd2e}

\def\emptyline{\vspace{12pt}}

\DefineParaStyle{Maple Heading 1}
\DefineParaStyle{Maple Text Output}
\DefineParaStyle{Maple Dash Item}
\DefineParaStyle{Maple Bullet Item}
\DefineParaStyle{Maple Normal}
\DefineParaStyle{Maple Heading 4}
\DefineParaStyle{Maple Heading 3}
\DefineParaStyle{Maple Heading 2}
\DefineParaStyle{Maple Warning}
\DefineParaStyle{Maple Title}
\DefineParaStyle{Maple Error}
\DefineCharStyle{Maple Hyperlink}
\DefineCharStyle{Maple 2D Math}
\DefineCharStyle{Maple Maple Input}
\DefineCharStyle{Maple 2D Output}
\DefineCharStyle{Maple 2D Input}

\title{Exam 2}
\author{Austin Gill}

\begin{document}
\maketitle

%% 1
\section{} \textit{When we talk about attacks on RSA, we usually talk about factoring $n$ into $pq$. But no one has been able to prove that this is the only way to find $d$ and $e$ given $n$. For example we could try to find $\varphi(n) = (p-1)(q-1)$ without first finding $p$ and $q$. Show that finding $\varphi(n)$ is sufficient to factor $n$ and therefore enough to find $d$.}

%% 2
\section{} \textit{Determine which of the following polynomials are irreducible over $\Z_2 [x]$.}

    \subsection{$x^5 + x^4 + 1$:}

    \subsection{$x^5 + x^3 + 1$:}

    \subsection{$x^5 + x^4 + x^2 + 1$:}

%% 3
\section{} \textit{My brother is color-blind, and we used to play snooker, if the balls had moved from their original positions he could not distinguished between a red and the green ball, as it is only the color that makes them non-identical. He was often skeptical that I was actually potting the balls in the correct order. I like to be able to prove to him that the two balls are in fact differently-colored. At the same time, I do not want him to learn which is red and which is green. Device a zero-knowledge protocol that allow me to prove that he really has two different colored balls in front of him. He is allowed to hold, move and handle the balls, I am only allowed to look at them. }

%% 4
\section{} \textit{Prove that an odd prime $p$ is expressible as a sum of two squares if and only if $p \equiv 1 \pmod{4}$.}

%% 5
\section{} \textit{A common way of storing passwords on a computer is to use DES with a password as the key to encrypt a fixed plaintext (often just $000\dots0$). The ciphertext is then stored in a file. When someone log in, the procedure is repeated and the ciphertexts are compared. Why is this a better method than using the password as the plaintext and a fixed key?}

%% 6
\section{} \textit{You have received the following message:}

    \begin{minted}[autogobble=false]{python}
        (949,   2750),  (8513,  28089), (5513,  8421),
        (4769,  4261),  (18352, 12856), (17914, 28599),
        (25231, 9196),  (3809,  5997),  (1477,  19626),
        (19108, 22326), (24966, 631),   (3494,  5974),
        (10256, 30308), (29093, 15082), (4223,  25106),
        (3595,  18546), (11325, 3588),  (5632,  4912),
        (18067, 13223), (21530, 3138),  (30949, 16065),
        (29784, 7987),  (6385,  5955),  (27338, 10405),
        (31715, 15969), (15815, 28055), (10462, 13371),
        (4852,  28393), (1331,  30788), (18117, 28680),
        (2472,  11786), (27548, 22909), (21980, 28433),
        (2154,  3440),  (21504, 22036), (13651, 18061),
        (10676, 26545), (30974, 23306), (14689, 8359)
    \end{minted}

\textit{It is an ElGamal ciphertext with the following parameters:}
    $$p = 31847$$
    $$\alpha = 5$$
    $$\beta = 18074$$
\textit{and your private random integer was}
    $$a = 7899$$
\textit{You also know that in order to translate the plaintext back into ordinary English text, you need to know how alphabetic characters were ``encoded'' as elements in $\Z_n$.  Each element of $\Z_n$ represent three alphabetic characters as in the following example:}

    \begin{align*}
        \text{DOG} &\to 3 \times 26^2 + 14 \times 26 + 6 = 2398\\
        \text{CAT} &\to 2 \times 26^2 + 0 \times 26 + 19 = 1371\\
        \text{ZZZ} &\to 25 \times 26^2 + 25 \times 26 + 25 = 17575\\
    \end{align*}

\textit{Decrypt the message (who sent it?), also explain why it is good that the first element in each pair is not the same.}

%% 7
\section{} \textit{Find all primes $p$ such that}

    \subsection{$p \mid 2^p + 1$}

    \subsection{$p \mid 2^p - 1$}

%% 8
\section{} \textit{I have five nieces and nephews, and I want to share a secret $(M)$ with them, and when three of them are in agreement they should be able to `unlock' it. I pick a prime $(p)$ larger than number of nieces and nephews and the secret number, $p = 17$. I calculate five pairs $(x_i, y_i)$ where $y_i \equiv M + s_1 x_i + s_2 x_i ^ 2\pmod{p}$, and $s_1, s_2$ are integers that only I know, and $x_1, \dots, x_5$ are distinct integers greater than $0$. Note that $f(0) \equiv M \pmod{p}$. I keep the polynomial secret, but I share $p$ and give each of them a $(x_i, y_i)$ pair. Three of them finally got together and agreed to try to solve my secret Lauren, Cohen and Kirian: $(1, 8)$, $(3, 10)$, and $(5, 11)$. The trouble is they can't agree on the math, so they ask you for help to solve this. Calculate the Lagrange Interpolating Polynomial and identify the secret $(M)$.}


\end{document}
