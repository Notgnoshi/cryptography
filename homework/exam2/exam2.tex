\documentclass[12pt]{article}
\usepackage{amsmath}
\usepackage{amssymb}
\usepackage{amsthm}
\usepackage{cancel}
\usepackage{color}
% Don't use the dumb paragraph indentation. For repeated breaks in mathematical chains the parskips
% suck.
\usepackage{parskip}
\usepackage[cache=false]{minted}
\usepackage{tikz}
\usepackage{graphicx}
\usepackage{float}
\usepackage[margin=1in]{geometry}
\usepackage[hidelinks]{hyperref}

\usepackage{maplestd2e}

\def\emptyline{\vspace{12pt}}

\DefineParaStyle{Maple Heading 1}
\DefineParaStyle{Maple Text Output}
\DefineParaStyle{Maple Dash Item}
\DefineParaStyle{Maple Bullet Item}
\DefineParaStyle{Maple Normal}
\DefineParaStyle{Maple Heading 4}
\DefineParaStyle{Maple Heading 3}
\DefineParaStyle{Maple Heading 2}
\DefineParaStyle{Maple Warning}
\DefineParaStyle{Maple Title}
\DefineParaStyle{Maple Error}
\DefineCharStyle{Maple Hyperlink}
\DefineCharStyle{Maple 2D Math}
\DefineCharStyle{Maple Maple Input}
\DefineCharStyle{Maple 2D Output}
\DefineCharStyle{Maple 2D Input}

\setminted{autogobble=true, style=tango, breaklines}
\usetikzlibrary{arrows, shapes, positioning}

\renewcommand{\thesubsection}{\thesection.\alph{subsection}}
% \renewcommand{\mod}[1]{\mathrm{mod}\ #1}
% \renewcommand{\pmod}[1]{(\mod{#1})}
\renewcommand{\qedsymbol}{$\blacksquare$}

\newcommand{\Z}{\mathbb{Z}}
\newtheorem*{defn}{Definition}
\newtheorem*{thm}{Theorem}

\newcommand{\cloud}[3]{
    \raisebox{-0.4\height}{
        \begin{tikzpicture}
            \node [cloud,
                   draw,
                   cloud puffs=#1,
                   cloud ignores aspect,
                   minimum height=#1,
                   minimum width=#3] {$\cdots$};
        \end{tikzpicture}
    }
}


% \usepackage{maplestd2e}
%
% \def\emptyline{\vspace{12pt}}
%
% \DefineParaStyle{Maple Heading 1}
% \DefineParaStyle{Maple Text Output}
% \DefineParaStyle{Maple Dash Item}
% \DefineParaStyle{Maple Bullet Item}
% \DefineParaStyle{Maple Normal}
% \DefineParaStyle{Maple Heading 4}
% \DefineParaStyle{Maple Heading 3}
% \DefineParaStyle{Maple Heading 2}
% \DefineParaStyle{Maple Warning}
% \DefineParaStyle{Maple Title}
% \DefineParaStyle{Maple Error}
% \DefineCharStyle{Maple Hyperlink}
% \DefineCharStyle{Maple 2D Math}
% \DefineCharStyle{Maple Maple Input}
% \DefineCharStyle{Maple 2D Output}
% \DefineCharStyle{Maple 2D Input}

\title{Exam 2}
\author{Austin Gill}

\begin{document}
\maketitle

%% 1
\section{} \textit{When we talk about attacks on RSA, we usually talk about factoring $n$ into $pq$. But no one has been able to prove that this is the only way to find $d$ and $e$ given $n$. For example we could try to find $\varphi(n) = (p-1)(q-1)$ without first finding $p$ and $q$. Show that finding $\varphi(n)$ is sufficient to factor $n$ and therefore enough to find $d$.}

    Since $n = pq$, we have that $\varphi(n) = (p - 1)(q - 1)$:

    \begin{align*}
        \varphi(n) &= (p - 1)(q - 1)\\
                   &= pq - p - q + 1\\
                   &= n - p - q + 1\\
                   &= (n + 1) - (p + q)\\
        \varphi(n) - (n + 1) &= - (p + q)\\
        (n + 1) - \varphi(n) &= p + q\\
    \end{align*}

    Giving that $q = (n + 1) - \varphi(n) - p$. But then we have

    \begin{align*}
        n &= pq\\
        n &= p \cdot \left((n + 1) - \varphi(n) - p\right)\\
    \end{align*}

    But this is a quadratic in $p$, which is easily solvable:

    \begin{align*}
        n &= -p^2 + p(n + 1 - \varphi(n))\\
        p^2 &- (n + 1 - \varphi(n))p + n = 0\\
    \end{align*}

    which may be solved using the quadratic formula with

    \begin{align*}
        a &= 1\\
        b &= -(n + 1 - \varphi(n))\\
        c &= n\\
    \end{align*}

    So then we have

    $$p = \frac{-b \pm \sqrt{b^2 - 4ac}}{2a}$$

    To test, we have

    \inputminted{python}{scripts/prob_1.py}

    which outputs

    \begin{minted}{python}
        (4811521585660301, 5473886613341039)
        (5473886613341039.0, 4811521585660301.0)
    \end{minted}

%% 2 -- Unfortunately, even though Maple has the functionality to solve these easily, I must still
%%      solve them by hand...
\section{} \textit{Determine which of the following polynomials are irreducible over $\Z_2 [x]$.}

    %% Factors into (x^2 + x + 1)(x^3 + x + 1)
    \subsection{$x^5 + x^4 + 1$:}
    %% Irreducible
    \subsection{$x^5 + x^3 + 1$:}
    %% Factors into (x + 1)(x^4 + x + 1)
    \subsection{$x^5 + x^4 + x^2 + 1$:}

        Being far too much work to do by hand during this hellish time of year, the following Python script runs a brute force trial division test on each of the above polynomials in $\Z_2[x]$.

        \inputminted{python}{scripts/prob_2abc.py}

        and produces the following output

        \begin{minted}{python}
        Running trial division on: x**5 + x**4 + 1
            Found divisors: Poly(x**2 + x + 1, x, modulus=2) Poly(x**3 + x + 1, x, modulus=2)
            Found divisors: Poly(x**3 + x + 1, x, modulus=2) Poly(x**2 + x + 1, x, modulus=2)
        Running trial division on: x**5 + x**3 + 1
        Running trial division on: x**5 + x**4 + x**2 + 1
            Found divisors: Poly(x + 1, x, modulus=2) Poly(x**4 + x + 1, x, modulus=2)
        \end{minted}

        Thus the first and last polynomials are reducible over $\Z_2[x]$, while the second polynomial is irreducible. Interesting note, Maple does not factor $x^5 + x^4 + x^2 + 1$, but my Python code above does.

    \textit{The field $GF(2^8)$ can be constructed as $\Z_2[x] \pmod{x^8 + x^4 + x^3 + x + 1}$}

    %% x^6 + x^5 + x^3 + x^2 + x
    \subsection{Compute ${(x^5 + 1)}^{-1}$:}

        This can be done in multiple ways: I could run through the Extended Euclidean Algorithm, or I could brute force check all of the elements of $\Z_2[x] \pmod{x^8 + x^4 + x^3 + x + 1}$ to see which one(s) if any produce the multiplicative identity. Since $\Z_2[x] \pmod{x^8 + x^4 + x^3 + x + 1}$ has only $2^8$ elements, the latter approach is what I choose.

        \inputminted{python}{scripts/prob_2d.py}
    %% x^7 + x^4 + x^3 + x^2
    \subsection{Compute $(x^4 + x^2) \times (x^3 + x + 1)$:}

        Sadly, this problem was faster to do by hand:

        $$(x^4 + x^2) \times (x^3 + x + 1)$$
        $$x^7 + 2x^5 + x^4 + x^3 + x^2$$

        But this is in $\Z_2[x]$, so we then have

        $$x^7 + x^4 + x^3 + x^2$$

        But now we are working $\pmod{x^8 + x^4 + x^3 + x + 1}$. Clearly\texttrademark{}, however, $x^7 + x^4 + x^3 + x^2 \equiv x^7 + x^4 + x^3 + x^2 \pmod{x^8 + x^4 + x^3 + x + 1}$. Reducing $\pmod{x^8 + x^4 + x^3 + x + 1}$ with the \texttt{crypto.math.reduce\textunderscore{}gf28} function I wrote confirms this.

        \begin{minted}{python}
            >>> from crypto.math import reduce_gf28
            >>> f = (x **4 + x**2) * (x**3 + x + 1)
            >>> reduce_gf28(f)
            Poly(x**7 + x**4 + x**3 + x**2, x, modulus=2)
        \end{minted}

    These functions have now been implemented in \texttt{crypto.math.polynomial} as the following:

    \begin{itemize}
        \item \texttt{reduce\textunderscore{}gf28(poly)}
        \item \texttt{coeffs2poly(coeffs)}
        \item \texttt{poly\textunderscore{}trial\textunderscore{}factor(poly)}
        \item \texttt{poly\textunderscore{}trial\textunderscore{}inverse(poly)}
    \end{itemize}

%% 3
\section{} \textit{My brother is color-blind, and we used to play snooker, if the balls had moved from their original positions he could not distinguished between a \red{} and the \green{} ball, as it is only the color that makes them non-identical. He was often skeptical that I was actually potting the balls in the correct order. I like to be able to prove to him that the two balls are in fact differently-colored. At the same time, I do not want him to learn which is \red{} and which is \green{}. Devise a zero-knowledge protocol that allows me to prove that he really has two different colored balls in front of him. He is allowed to hold, move and handle the balls, I am only allowed to look at them.}

    Instruct your brother to hide the balls and repeatedly reveal them to you in random order. He will quiz you on which one is what color. You may tell the truth or not, but you must do so consistently.

    Your brother will not know which one is \green{} and which one is \red{}, only that you consistently label the same ball with the same color. Over time this probabilistically shows that you can tell the difference between the two balls.

    \begin{enumerate}
        \item[\textbf{Case 1:}] Both balls are the same color. You will not be able to tell the difference between them, so when your brother shows them to you in random order, you cannot possibly correctly label one ball as \red{} and one ball as \green{} consistently.

        \item[\textbf{Case 2:}] One ball is \red{} and one ball is \green{}. You are now able to differentiate between the \red{} and \green{} ball consistently when they are showed to you in random order, even if you choose to label them as {\color{red}green} and {\color{green}red}.

        \item[\textbf{Case 3:}] One ball is \purple{} and one ball is \orange{}. You are still able to differentiate between the two balls consistently. This is still enough to convince a rational person that they are in fact differently colored.
    \end{enumerate}

%% 4 -- quite hard
\section{} \textit{Prove that an odd prime $p$ is expressible as a sum of two squares if and only if $p \equiv 1 \pmod{4}$.}

    \begin{description}
        \item[$(\Rightarrow)$]
        \begin{proof}
            Suppose $p = x^2 + y^2$ is prime. The elements of $\Z_4$ are $\{0, 1, 2, 3\}$. Thus the elements of $\{x^2 \mid x \in \Z_4\}$ are $\{0, 1, 4, 9\} = \{0, 1\}$. Thus $x^2 \pmod{4}$ can only take on the values $0$ or $1$. Thus if a prime $p = x^2 + y^2$, it must take on one of the values in $\{0 + 0, 0 + 1, 1 + 0, 1+ 1\} = \{0, 1, 2\}$

            However, $p$ cannot be congruent to $0 \pmod{4}$ because it would not be prime, and likewise $p$ cannot be congruent to $2 \pmod{4}$ save for $p = 2$ for the same reason. Thus the only option left is for $p \equiv 1 \pmod{4}$.
        \end{proof}

        \item[$(\Leftarrow)$]
        \begin{proof}
            Suppose $p \equiv 1 \pmod{4}$ is prime. This direction however, isn't nearly as trivial, and relies on an intermediate result.

            \begin{lemma}
                If $p \equiv 1 \pmod{4}$ is prime, then there exists a solution $a$ to $a^2 \equiv -1 \pmod{p}$.
            \end{lemma}

            \begin{proof}
                Consider the Legendre Symbol $\legendre{-1}{p} = {(-1)}^{\frac{p - 1}{2}} = {(-1)}^{2k} = 1$, indicating the presence of such a solution $a$ to $a^2 \equiv -1 \pmod{p}$.
            \end{proof}

            Let $a$ be a solution to $a^2 \equiv -1 \pmod{p}$, and let $k = \floor{\sqrt{p}}$. Note then that $k \leq \sqrt{p} \geq k + 1$. Now consider the numbers $r - sa$ for $r, s$ integers between $0$ and $k$ inclusive. From combinatorics, there are ${(k + 1)}^2$ combinations.

            Since ${(k + 1)}^2 > p$, there must be at least two distinct such pairs $(r, s)$ and $(t, u)$ where $r - sa$ and $t - ua$ are congruent $\pmod{p}$.

            $$r - sa \equiv t - ua \pmod{p}$$

            Now define $x = r - s$ and $y = t - u$ and note that both $x$ and $y$ are not both $0$ with the following relations holding:

            $$\vert x \vert \leq k$$
            $$\vert y \vert \leq k$$
            $$x \equiv ya \pmod{p}$$

            Therefore $x^2 \equiv y^2 a^2 \equiv -y^2 \pmod{p}$, and thus $p$ divides $x^2 + y^2$. However, note that $0 < x^2 + y^2 \leq 2k^2 < 2p$, which forces $p = x^2 + y^2$.
        \end{proof}
    \end{description}

%% 5
\section{} \textit{A common way of storing passwords on a computer is to use DES with a password as the key to encrypt a fixed plaintext (often just $000\dots0$). The ciphertext is then stored in a file. When someone log in, the procedure is repeated and the ciphertexts are compared. Why is this a better method than using the password as the plaintext and a fixed key?}

    I assume that there is one file for each password, and that if the password is used as the plaintext, the key for each file is different.

    If the key is fixed, and the password is stored as plaintext, then should the key be found, the security is compromised forever. Should the plaintext be fixed and the password used as the key, and the password is found, then security is compromised only as long as the password remains unchanged.

    Further, access can be granted without the system software ever comparing the passwords. All that is needed is to input the password attempt as a potential key, then compare ciphertexts. This is desirable, as the actual and potential passwords are never compared.

    In the other system, a user inputs a potential password, and the system decrypts the stored ciphertext with a key it knows, and compares the actual password and the potential password. This is extremely undesirable. It loads the actual password into memory, which is a risk in and of itself. Further, the comparison of the potential and actual password could be suceptible to timing and power attacks.

%% 6
\section{} \textit{You have received the following message:}

    \begin{minted}[autogobble=false]{python}
        (949,   2750),  (8513,  28089), (5513,  8421),
        ...
        (10676, 26545), (30974, 23306), (14689, 8359)
    \end{minted}

    \textit{It is an ElGamal ciphertext with the following parameters:}
        $$p = 31847$$
        $$\alpha = 5$$
        $$\beta = 18074$$
    \textit{and your private random integer was}
        $$a = 7899$$
    \textit{You also know that in order to translate the plaintext back into ordinary English text, you need to know how alphabetic characters were ``encoded'' as elements in $\Z_n$.  Each element of $\Z_n$ represent three alphabetic characters as in the following example:}

        \begin{align*}
            \text{DOG} &\to 3  \times 26^2 + 14 \times 26 + 6  = 2398\\
            \text{CAT} &\to 2  \times 26^2 + 0  \times 26 + 19 = 1371\\
            \text{ZZZ} &\to 25 \times 26^2 + 25 \times 26 + 25 = 17575\\
        \end{align*}

    \textit{Decrypt the message (who sent it?), also explain why it is good that the first element in each pair is not the same.}

    The following Python script

    \inputminted{python}{scripts/prob_6.py}

    produces the following snippet of the Ring Verse from The Lord of the Rings

    \begin{quote}
        \centering
        \textit{One Ring to rule them all, One Ring to find them,}\\
        \textit{One Ring to bring them all, and in the darkness bind them,}\\
        \textit{In the Land of Mordor where the Shadows lie.}\\
    \end{quote}

    The fact that the last line is not often quoted, and the fact that the message was transmitted as an ElGamal encrypted message leads me to believe that a nerd/geek/professor sent the message. Further, the message was likely transmitted in order to demonstrate nerdom and/or to test the recipient's capabilities. I propose that in light of these likelihoods, the transmitter be named Alice.

    The first element in each pair is the number $r \equiv \alpha^k \pmod{p}$. If $r$ differs for each pair, then $k$, the secret random integer Alice generates also must differ for each pair. It is important ensure a different random $k$ is used for each message because an attacker would then have to solve a discrete logarithm problem for each message $m_i$ sent. This is untenable for large amounts of messages.

%% 7
\section{} \textit{Find all primes $p$ such that}

    \subsection{$p \mid 2^p + 1$}

        \begin{align*}
            p &\mid 2^p + 1\\
            pk &= 2^p + 1 &\text{for some $k$}\\
            2^p &= pk - 1\\
            2^p &\equiv -1 \pmod{p}\\
        \end{align*}

        But by Fermat's Little Theorem, we have

        \begin{align*}
            a^{p - 1} &\equiv 1 \pmod{p}\\
            a^p &\equiv a \pmod{p}\\
            2^p &\equiv 2 \pmod{p}\\
        \end{align*}

        Thus $-1 \equiv 2 \pmod{p}$, giving that $-1 = pk + 2$, or $-pk = 3$, which only has integer solutions with $p$ prime of $p=3$ and $k=-1$

    \subsection{$p \mid 2^p - 1$}

        Similarly,

        \begin{align*}
            p &\mid 2^p - 1\\
            pk &= 2^p - 1 &\text{for some $k$}\\
            2^p &= pk + 1\\
            2^p &\equiv 1 \pmod{p}\\
        \end{align*}

        And again by Fermat's Little Theorem, we still have $2^p \equiv 2 \pmod{p}$. Thus $1 \equiv 2 \pmod{p}$, or $1 = pk + 2$ giving that $-1 = pk$ which has no integer solutions for a prime $p$.

%% 8
\section{} \textit{I have five nieces and nephews, and I want to share a secret $(M)$ with them, and when three of them are in agreement they should be able to `unlock' it. I pick a prime $(p)$ larger than number of nieces and nephews and the secret number, $p = 17$. I calculate five pairs $(x_i, y_i)$ where $y_i \equiv M + s_1 x_i + s_2 x_i ^ 2\pmod{p}$, and $s_1, s_2$ are integers that only I know, and $x_1, \dots, x_5$ are distinct integers greater than $0$. Note that $f(0) \equiv M \pmod{p}$. I keep the polynomial secret, but I share $p$ and give each of them a $(x_i, y_i)$ pair. Three of them finally got together and agreed to try to solve my secret Lauren, Cohen and Kirian: (1, 8), (3, 10), and (5, 11). The trouble is they can't agree on the math, so they ask you for help to solve this. Calculate the Lagrange Interpolating Polynomial and identify the secret $(M)$.}

    We have the following:

    \begin{align*}
        l_k(x) &\equiv \prod_{\substack{i = 1 \\ i \neq k}}^3 \frac{x - x_i}{x_k - x_i} &\pmod{17}\\
        p(x) &\equiv \sum_{k = 1}^3 y_k l_k(x) &\pmod{17}
    \end{align*}

    Thus, for the ordered pairs (1, 8), (3, 10), and (5, 11), we have

    \begin{align*}
        p(x) &\equiv 8l_1(x) + 10l_2(x) + 11l_3(x) &\pmod{17}\\
        p(x) &\equiv 8\left(\frac{x - x_2}{x_1 - x_2} \cdot \frac{x - x_3}{x_1 - x_3}\right) + 10\left(\frac{x - x_1}{x_2 - x_1} \cdot \frac{x - x_3}{x_2 - x_3}\right) + 11\left(\frac{x - x_1}{x_3 - x_1} \cdot \frac{x - x_2}{x_3 - x_2}\right) &\pmod{17}\\
        p(x) &\equiv 8\left(\frac{x - 3}{1 - 3} \cdot \frac{x - 5}{1 - 5}\right) + 10\left(\frac{x - 1}{3 - 1} \cdot \frac{x - 5}{3 - 5}\right) + 11\left(\frac{x - 1}{5 - 1} \cdot \frac{x - 3}{5 - 3}\right) &\pmod{17}\\
        p(x) &\equiv 8\left(\frac{x - 3}{-2} \cdot \frac{x - 5}{-4}\right) + 10\left(\frac{x - 1}{2} \cdot \frac{x - 5}{-2}\right) + 11\left(\frac{x - 1}{4} \cdot \frac{x - 3}{2}\right) &\pmod{17}\\
        p(x) &\equiv \frac{8}{8} \cdot (x - 3) \cdot (x - 5) + \frac{10}{-4} \cdot (x - 1) \cdot (x - 5) + \frac{11}{8} \cdot (x - 1) \cdot (x - 3) &\pmod{17}\\
        p(x) &\equiv 1 \cdot (x - 3) \cdot (x - 5) + 6 \cdot (x - 1) \cdot (x - 5) + 12 \cdot (x - 1) \cdot (x - 3) &\pmod{17}\\
    \end{align*}

    But $M \equiv p(0) \pmod{p}$, so

    \begin{align*}
        p(0) &\equiv 1 \cdot (-3) \cdot (-5) + 6 \cdot (-1) \cdot (-5) + 12 \cdot (-1) \cdot (-3) &\pmod{17}\\
        p(0) &\equiv 15 + 30 + 36 &\pmod{17}\\
        M \equiv p(0) &\equiv 13 &\pmod{17}\\
    \end{align*}

    Verifying, we have:

    \begin{minted}{python}
        >>> from sympy import Symbol
        >>> x = Symbol('x')
        >>> p = 1 * (x - 3) * (x - 5) + 6 * (x - 1) * (x - 5) + 12 * (x - 1) * (x - 3)
        >>> p.as_poly(x, modulus=17)
        Poly(2*x**2 - 7*x - 4, x, modulus=17)
        >>> -7 % 17
        10
        >>> -4 % 17
        13
        >>> def f(x):
        ...     return (2 * x**2 + 10 * x + 13) % 17
        >>> f(1)
        8
        >>> f(3)
        10
        >>> f(5)
        11
    \end{minted}
\end{document}
