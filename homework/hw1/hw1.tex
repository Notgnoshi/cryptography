\documentclass[12pt]{article}
\usepackage{amsmath, amssymb, amsthm, minted}
\usepackage[margin=1in]{geometry}
\usepackage[hidelinks]{hyperref}


\renewcommand{\mod}[1]{\mathrm{mod}\ #1}
\renewcommand{\pmod}[1]{\ (\mod{#1})}
\newtheorem*{defn}{Definition}

\title{Homework 1}
\author{Austin Gill}

\begin{document}
\maketitle

Note. My \texttt{crypto} Python library used below can be found at \url{https://github.com/Notgnoshi/cryptography}, as can several Matlab snippets.
\section*{(1)} \textit{Suppose you encrypt a text using an affine cipher, then encrypt the encryption using another affine cipher (both are working $\mod{26}$). Is there any advantage to doing this, rather than using a single affine cipher? Why or why not?}

\begin{defn}
Let $\alpha , \beta \in \mathbb Z$ with $\gcd(\alpha , 26) = 1$. The function $E(m) = \alpha m_i + \beta \pmod{26}$ is called an affine cipher.
\end{defn}

Suppose $E(m) = (F \circ G)(m)$, where $F(m) = am_i + b$ and $G(m) = cm_i + d$ are both affine ciphers $\pmod{26}$. Then

\begin{align*}
    E(m) &= F(G(m))\\
         &= a(cm_i + d) + b \pmod{26}\\
         &= acm_i + ad + b \pmod{26}\\
         &= pm_i + q \pmod{26}\\
\end{align*}

where $p = ac \pmod{26}$ and $q = ad + b \pmod{26}$, which is no different than a single affine cipher. The difficulty of cracking this cipher remains unchanged.

\section*{(2)} \textit{Suppose we work $\mod{27}$ instead of $\mod{26}$ for affine ciphers. How many keys are possible? What if we work $\mod{29}$?}

The following numbers are those relatively prime with 26, 27, and 29.
\begin{itemize}
  \item[$\pmod{26}$:]
  \texttt{[1, 3, 5, 7, 9, 11, 15, 17, 19, 21, 23, 25]}
  \item[$\pmod{27}$:]
  \texttt{[1, 2, 4, 5, 7, 8, 10, 11, 13, 14, 16, 17, 19, 20, 22, 23, 25, 26]}
  \item[$\pmod{29}$:]
  \texttt{[1, 2, 3, 4, 5, 6, 7, 8, 9, 10, 11, 12, 13, 14, 15, 16, 17, 18, 19, 20, 21, 22, 23, 24, 25, 26, 27, 28]}
\end{itemize}
There are 12 numbers relatively prime with 26, 18 with 27, and 28 with 29. In general, if there are $m$ numbers relatively prime with $n$, then there are $n\cdot m$ key possibilities for an affine cipher.

The following are the number of key possibilities for 26, 27, and 29.

\begin{itemize}
  \item[$\pmod{26}$:] 312 key possibilities
  \item[$\pmod{27}$:] 486 key possibilities
  \item[$\pmod{29}$:] 812 key possibilities
\end{itemize}

\section*{(3)} \textit{Eve captures Bob’s Hill cipher machine, which uses a $2\times2$ matrix $M \mod{26}$. She tries a chosen plaintext attack. She finds that the plaintext $ba$ encrypts to $HC$ and the plaintext $zz$ encrypts to $GT$. What is the matrix $M$.}

Solve the matrix equation

$$\begin{pmatrix}1 & 0\\25 & 25\end{pmatrix}\begin{pmatrix}a & b\\c & d\end{pmatrix} \equiv \begin{pmatrix}7 & 2\\6 & 19\end{pmatrix} \pmod{26}$$

Note that oddly enough, $\begin{pmatrix}1 & 0\\25 & 25\end{pmatrix}$ is its own modular inverse. So, we have

$$M = \begin{pmatrix}a & b\\c & d\end{pmatrix} = \begin{pmatrix}1 & 0\\25 & 25\end{pmatrix}\begin{pmatrix}7 & 2\\6 & 19\end{pmatrix} \pmod{26} = \begin{pmatrix}7 & 2\\12 & 5\end{pmatrix}$$

\section*{(4)} \textit{A sequence generated by a length three recurrence starts $001110$. Find the next four elements of the sequence.}

Knowing the length of the recurrence, set up and solve the following matrix equation for $\vec{c}$

$$\begin{pmatrix}x_1 & x_2 & \cdots & x_m \\ x_2 & x_2 & \cdots & x_{m+1} \\ \vdots & \vdots & \ddots & \vdots \\ x_m & x_{m+1} & \cdots & x_{2m - 1}\end{pmatrix} \cdot \vec{c} \equiv \begin{pmatrix}x_{m+1} \\ x_{m+2} \\ \vdots \\ x_{2m}\end{pmatrix}$$

where the recurrence is

$$x_m = \begin{pmatrix}x_{n} \\ x_{n + 1} \\ \vdots \\ x_{m - 1}\end{pmatrix} \cdot \vec{c}^T$$

In this particular case, we have the following

$$\begin{pmatrix}0 & 0 & 1 \\ 0 & 1 & 1 \\ 1 & 1 & 1\end{pmatrix}\cdot\begin{pmatrix}c_0 \\ c_1 \\ c_2\end{pmatrix} = \begin{pmatrix}1 \\ 1 \\ 0\end{pmatrix}$$

which yields

$$\vec{c} = \begin{pmatrix}-1 \\ 0 \\ 1\end{pmatrix} \equiv \begin{pmatrix}1 \\ 0 \\ 1\end{pmatrix} \pmod{2}$$

resulting in the recurrence

$$x_{n + 3} = x_n + x_{n + 2}$$

which implemented in Matlab yields the following

\begin{minted}{matlab}
function seq = recurrence(x, c, seq_length)
    [m, n] = size(c);
    for i = 1:seq_length
        x(i + n) = mod(dot(c, x(i : i+n-1)), 2);
    end

    seq = x;
end

>>> x = [0 0 1];
>>> c = [1 0 1];
>>> recurrence(x, c, 12)
    ans =
        0 0 1 1 1 0 1 0 0 1 1 1 0 1 0
\end{minted}

\section*{(5)} \textit{Experiment with the affine cipher $y = m x + n \pmod{26}$ for values of $m > 26$. In particular, determine whether or not these encryptions are the same as ones obtained with $m < 26$.}

The following Python script verifies that the encryptions do not differ for $n < 26$ and $m > 26$ as long as $n \equiv m \pmod{26}$.

\begin{minted}{python}
#!/usr/bin/python3
from crypto.classical import AffineCipher
import math
import random

def main():
    message = 'thisisatest'
    # Generate lists of numbers relatively prime with 26.
    lesser_nums = filter(lambda x: math.gcd(x, 26) == 1, range(0, 27))
    greater_nums = filter(lambda x: math.gcd(x, 26) == 1, range(26, 26 * 2))

    # Pairwise iterate over lesser and greater numbers.
    for l, g in zip(lesser_nums, greater_nums):
        shift = random.randrange(0, 27)
        lesser = AffineCipher(l, shift)
        greater = AffineCipher(g, shift)

        l_cipher = lesser.encrypt(message)
        g_cipher = greater.encrypt(message)
        print('lesser (m={}): {}'.format(l, l_cipher))
        print('greater (m={}): {}'.format(g, g_cipher))

if __name__ == '__main__':
    main()
\end{minted}

In particular, we have that

\begin{align*}
    a \cdot b \pmod{m} &= a \pmod{m} \cdot b \pmod{m}\\
    a + b \pmod{m} &= a \pmod{m} + b \pmod{m}\\
\end{align*}

\section*{(6)} \textit{The following sequence was generated by a linear feedback shift register. Determine the recurrence that generated it.}

\texttt{[1, 0, 1, 0, 0, 1, 1, 0, 1, 1, 0, 0, 0, 1, 0, 0, 1, 0, 0, 0, 0, 1, 1,
1, 0, 0, 0, 0, 0, 1, 0, 1, 1, 1, 1, 1, 1, 0, 0, 1, 0, 1, 0, 1, 0, 0,
0, 1, 1, 0, 0, 1, 1, 1, 1, 0, 1, 1, 1, 0, 1, 0, 1, 1, 0, 1, 0, 0, 1,
1, 0, 1, 1, 0, 0, 0, 1, 0, 0, 1, 0, 0, 0, 0, 1, 1, 1, 0, 0, 0]}

Using the following Matlab code

\begin{minted}{matlab}
for size = 1:(length(x) / 2)
    A = gen_matrix(x, size);
    d = mod(round(det(A)), 2);

    fprintf('size: %d det: %d (mod 2)\n', size, d);

    if d ~= 0
        b = x(size + 1 : 2 * size)';
        A_inv = matmodinverse(A, 2);
        c = mod(A_inv * b, 2)';

        if isequal(x, recurrence(x, c, length(x) - size))
           fprintf('sequences equal\n');
           disp(c);
        end
    end
end
\end{minted}

Which uses the following two functions

\begin{minted}{matlab}
function A = gen_matrix(seq, size)
    A = [];
    for i = 1:size
        A = [A; seq(i:i + size - 1)];
    end
end

function Ainv = matmodinverse(A, m)
    a = round(det(A));
    res = mod(a * (1:m), m);
    b = find(res == 1);
    Ainv = mod(b * round(a * inv(A)), m);
end
\end{minted}

the matrix size was determined to be $6$, and the vector $\vec{c} = \begin{pmatrix}1 & 1 & 0 & 1 & 1 & 0\end{pmatrix}$ yielding the recurrence relation

$$ x_{n + 6} = x_n + x_{n + 1} + x_{n + 3} + x_{n + 4}$$

\section*{(7)} \textit{You have caught the following ciphertext, and you believe it was encrypted by an affine cipher:}

\begin{center}
\texttt{edsgickxhuklzveqzvkxwkzukcvuh}
\end{center}

\noindent\textit{You also believe that the first two letters of the plaintext are} \texttt{if}. \textit{Decrypt.}

That is,

$$\begin{cases}
a \cdot ``i" + b = ``e" \pmod{26}\\
a \cdot ``f" + b = ``d" \pmod{26}\\
\end{cases}$$

or rather,

$$\begin{cases}
8a + b = 4 \pmod{26}\\
5a + b = 3 \pmod{26}\\
\end{cases}$$

Yielding the solution $a = 9$ and $b = 10$. The following Python snippet confirms

\begin{minted}{matlab}
>>> from crypto.classical import AffineCipher
>>> cipher = 'edsgickxhuklzveqzvkxwkzukcvuh'
>>> affine = AffineCipher(a=9, b=10)
>>> affine.encrypt('if')
'ed'
>>> affine.decrypt(cipher)
'ifyoucanreadthisthankateacher'
\end{minted}

\end{document}
