\documentclass[12pt]{article}
\usepackage{amsmath}
\usepackage{amssymb}
\usepackage{amsthm}
\usepackage{cancel}
\usepackage{color}
% Don't use the dumb paragraph indentation. For repeated breaks in mathematical chains the parskips
% suck.
\usepackage{parskip}
\usepackage[cache=false]{minted}
\usepackage{tikz}
\usepackage{graphicx}
\usepackage{float}
\usepackage[margin=1in]{geometry}
\usepackage[hidelinks]{hyperref}

\usepackage{maplestd2e}

\def\emptyline{\vspace{12pt}}

\DefineParaStyle{Maple Heading 1}
\DefineParaStyle{Maple Text Output}
\DefineParaStyle{Maple Dash Item}
\DefineParaStyle{Maple Bullet Item}
\DefineParaStyle{Maple Normal}
\DefineParaStyle{Maple Heading 4}
\DefineParaStyle{Maple Heading 3}
\DefineParaStyle{Maple Heading 2}
\DefineParaStyle{Maple Warning}
\DefineParaStyle{Maple Title}
\DefineParaStyle{Maple Error}
\DefineCharStyle{Maple Hyperlink}
\DefineCharStyle{Maple 2D Math}
\DefineCharStyle{Maple Maple Input}
\DefineCharStyle{Maple 2D Output}
\DefineCharStyle{Maple 2D Input}

\setminted{autogobble=true, style=tango, breaklines}
\usetikzlibrary{arrows, shapes, positioning}

\renewcommand{\thesubsection}{\thesection.\alph{subsection}}
% \renewcommand{\mod}[1]{\mathrm{mod}\ #1}
% \renewcommand{\pmod}[1]{(\mod{#1})}
\renewcommand{\qedsymbol}{$\blacksquare$}

\newcommand{\Z}{\mathbb{Z}}
\newtheorem*{defn}{Definition}
\newtheorem*{thm}{Theorem}

\newcommand{\cloud}[3]{
    \raisebox{-0.4\height}{
        \begin{tikzpicture}
            \node [cloud,
                   draw,
                   cloud puffs=#1,
                   cloud ignores aspect,
                   minimum height=#1,
                   minimum width=#3] {$\cdots$};
        \end{tikzpicture}
    }
}


\title{Homework 2}
\author{Austin Gill}

\begin{document}
\maketitle

\section{} \textit{Naive Nelson uses RSA to receive a single ciphertext $c$, corresponding to the message $m$. His public modulus is $n$ and his public encryption exponent is $e$. Since he feels guilty that his system was used only once, he agrees to decrypt any ciphertext $t$ that someone sends him, as long as it is not $c$, and return the answer to that person.  Evil Eve sends him the ciphertext ${2^e}c \pmod{n}$. Show how this allows Eve to find $m$.}

\section{} \textit{In order to increase security, Bob chooses $n$ and two encryption exponents $e_1$, $e_2$. He asks Alice to encrypt her message $m$ to him by first computing $c_1 \equiv m^{e_1} \pmod{n}$, then encrypting $c_1$ to get $c_2 \equiv c_1^{e_1} \pmod{n}$. Alice  then sends $c_1$ to Bob.  Does this double encryption increase security over single encryption?  Why or why not?}

\section{} \textit{Show that if $p$ is prime and $a$ and $b$ are integers not divisible by $p$ with $a^p \equiv b^b \pmod{p}$, then $a^p \equiv b^b \pmod{p^2}$}

\section{} \textit{Your opponent uses RSA with $n = pq$ and encryption exponent $e$ and encrypts a message $m$. This yields the ciphertext $c \equiv m^e \pmod{n}$. A spy tells you that, for this message, $m^{12345} \equiv 1 \pmod{n}$. Describe how to determine $m$. Note that you do not know $p$, $q$, $\varphi(n)$, or the secret decryption exponent $d$. However, you should find a decryption exponent that works for this particular ciphertext. Moreover, explain carefully why your decryption works (your explanation must include how the spy’s information is used)}

\section{}
    \subsection{}\label{5a} \textit{Show that the last two decimal digits of a perfect square must be one of the following pairs: $\{00, e1, e4, 25, o6, e9\}$ where $e$ is any even digit and $o$ is any odd digit. (Hint: Show that $n^2$, ${(50 + n)}^2$, and ${(50 - n)}^2$ all have the same final decimal digits and then consider those integers $n$ with $0 \leq n \leq 25$)}

    \subsection{} \textit{Explain how the result of part~\ref{5a} can be used to speed up Fermat's factorization method.}
\end{document}
