\documentclass[12pt]{article}
\usepackage{amsmath}
\usepackage{amssymb}
\usepackage{amsthm}
\usepackage{cancel}
\usepackage{color}
% Don't use the dumb paragraph indentation. For repeated breaks in mathematical chains the parskips
% suck.
\usepackage{parskip}
\usepackage[cache=false]{minted}
\usepackage{tikz}
\usepackage{graphicx}
\usepackage{float}
\usepackage[margin=1in]{geometry}
\usepackage[hidelinks]{hyperref}

\usepackage{maplestd2e}

\def\emptyline{\vspace{12pt}}

\DefineParaStyle{Maple Heading 1}
\DefineParaStyle{Maple Text Output}
\DefineParaStyle{Maple Dash Item}
\DefineParaStyle{Maple Bullet Item}
\DefineParaStyle{Maple Normal}
\DefineParaStyle{Maple Heading 4}
\DefineParaStyle{Maple Heading 3}
\DefineParaStyle{Maple Heading 2}
\DefineParaStyle{Maple Warning}
\DefineParaStyle{Maple Title}
\DefineParaStyle{Maple Error}
\DefineCharStyle{Maple Hyperlink}
\DefineCharStyle{Maple 2D Math}
\DefineCharStyle{Maple Maple Input}
\DefineCharStyle{Maple 2D Output}
\DefineCharStyle{Maple 2D Input}

\setminted{autogobble=true, style=tango, breaklines}
\usetikzlibrary{arrows, shapes, positioning}

\renewcommand{\thesubsection}{\thesection.\alph{subsection}}
% \renewcommand{\mod}[1]{\mathrm{mod}\ #1}
% \renewcommand{\pmod}[1]{(\mod{#1})}
\renewcommand{\qedsymbol}{$\blacksquare$}

\newcommand{\Z}{\mathbb{Z}}
\newtheorem*{defn}{Definition}
\newtheorem*{thm}{Theorem}

\newcommand{\cloud}[3]{
    \raisebox{-0.4\height}{
        \begin{tikzpicture}
            \node [cloud,
                   draw,
                   cloud puffs=#1,
                   cloud ignores aspect,
                   minimum height=#1,
                   minimum width=#3] {$\cdots$};
        \end{tikzpicture}
    }
}


\title{Homework 2}
\author{Austin Gill}

\begin{document}
\maketitle

\section{} \textit{Naive Nelson uses RSA to receive a single ciphertext $c$, corresponding to the message $m$. His public modulus is $n$ and his public encryption exponent is $e$. Since he feels guilty that his system was used only once, he agrees to decrypt any ciphertext $t$ that someone sends him, as long as it is not $c$, and return the answer to that person.  Evil Eve sends him the ciphertext ${2^e}c \pmod{n}$. Show how this allows Eve to find $m$.}

    Naive Nelson decrypts by raising the ciphertext to the $d$th power $\pmod{n}$, with $de \equiv 1 \pmod{n}$. Note also that $2^e c \equiv 2^e m^e \equiv {2m}^e \pmod{n}$. Thus, to decrypt Eve's ciphertext, Nelson does the following

    $${(2^e c)}^d \equiv {(2m)}^{de} \equiv {(2m)}^1 \equiv 2m \pmod{n}$$

    from which it is a trivial matter to recover $m$.

\section{} \textit{In order to increase security, Bob chooses $n$ and two encryption exponents $e_1$, $e_2$. He asks Alice to encrypt her message $m$ to him by first computing $c_1 \equiv m^{e_1} \pmod{n}$, then encrypting $c_1$ to get $c_2 \equiv c_1^{e_1} \pmod{n}$. Alice  then sends $c_1$ to Bob.  Does this double encryption increase security over single encryption?  Why or why not?}

    One of the attacks on RSA deals with factoring $n$, which double encryption has no impact on.

    Note also that this double encryption is equivalent to computing $c_2 = {(m^{e_1})}^{e_2} \equiv m^{e_1 e_2} \equiv m^e \pmod{n}$ for $e = e_1 e_2$. Then to decrypt, use the decryption exponent $d = d_1d_2$.

\section{} \textit{Show that if $p$ is prime and $a$ and $b$ are integers not divisible by $p$ with $a^p \equiv b^b \pmod{p}$, then $a^p \equiv b^p \pmod{p^2}$}

    By Fermat's Little Theorem we have $a^{p - 1} \equiv 1 \pmod{p}$ and $b^{p - 1} \equiv 1 \pmod{p}$, so then

    \begin{align*}
        a \cdot a ^ {p - 1} &\equiv a \cdot 1 &\pmod{p}\\
        b \cdot b ^ {p - 1} &\equiv b \cdot 1 &\pmod{p}\\
        a^p &\equiv a &\pmod{p}\\
        b^p &\equiv b &\pmod{p}\\
    \end{align*}

    Thus $a \equiv b \pmod{p}$, or rather $a = b + pk$ for some integer $k$. But then

    \begin{align*}
        a &= b + pk\\
        a^p &= {(b + pk)}^p\\
            &= \sum_{j = 0}^{p} \binom{p}{j} {(kp)}^j b^{p - j} &\text{(binomial theorem)}\\
            &= \binom{p}{0}b^p + \binom{p}{1}(kp)b^{p - 1} + \binom{p}{2}{(kp)}^2 b^{p - 2} \cdots + \binom{p}{p}{(kp)}^p\\
            &= b^p + kp^2b^{p - 1} + {(kp)}^2 \frac{p(p - 1)}{2} b^{p - 2} + \cdots + {(kp)}^p\\
            &= b^p + p^2 \cdot \left(\cloud{32.7}{2cm}{7cm}\right) &\text{(factor out $p^2$)}\\
    \end{align*}

    Thus $a^p$ may be written as $b^p + p^2 \left(\cloud{15.4}{1.5cm}{2cm}\right)$, where \cloud{15.4}{1.5cm}{2cm} is some integer whose value I don't care about. Therefore $a^p \equiv b^p \pmod{p}$.

\section{} \textit{Your opponent uses RSA with $n = pq$ and encryption exponent $e$ and encrypts a message $m$. This yields the ciphertext $c \equiv m^e \pmod{n}$. A spy tells you that, for this message, $m^{12345} \equiv 1 \pmod{n}$. Describe how to determine $m$. Note that you do not know $p$, $q$, $\varphi(n)$, or the secret decryption exponent $d$. However, you should find a decryption exponent that works for this particular ciphertext. Moreover, explain carefully why your decryption works (your explanation must include how the spy's information is used)}

    The book's prompt states to assume $\gcd(12345, e) = 1$. Thus we can use the extended Euclidean algorithm to find the multiplicative inverse of $d^{-1} = e \pmod{12345}$. Then write $e d^{-1} = 12345k + 1$ for some $k \in \Z$, which gives us ${(m^e)}^{d^{-1}} \equiv m^{12345k + 1} \equiv m \cdot m^{12345k} \equiv m \cdot {(m^{12345})}^k \equiv m 1^k \equiv m \pmod{n}$.

    Another option is to relate $\varphi(n)$ and 12345 using Euler's Theorem:

    $$m^{\varphi(n)} \equiv 1 \pmod{n}$$
    $${(m^k)}^{\varphi(n)} \equiv 1 \pmod{n}$$
    $${(m^{\varphi(n)})}^{k} \equiv 1 \pmod{n}$$

    Thus either $\varphi(n)$ is a multiple of 12345, or 12345 is a multiple of $\varphi(n)$. In either case, this allows us to find $\varphi(n)$ relatively easily, at which point finding $m$ is trivial. However, $12345 = 3 * 5 * 823$

    \begin{minted}{python}
        from crypto.math import factor
        factor(12345, 'fermat')
    \end{minted}

    gives

    \begin{minted}{python}
        [5, 3, 823]
    \end{minted}

    But $\varphi(n)$ must be even, so we know $\varphi(n) \nmid 12345$.

\section{}
    \subsection{} \textit{Show that the last two decimal digits of a perfect square must be one of the following pairs: $\{00, e1, e4, 25, o6, e9\}$ where $e$ is any even digit and $o$ is any odd digit. (Hint: Show that $n^2$, ${(50 + n)}^2$, and ${(50 - n)}^2$ all have the same final decimal digits and then consider those integers $n$ with $0 \leq n \leq 25$)}

    \begin{align*}
        n^2 &= &n^2\\
        {(50 + n)}^2 &= \underbrace{2500 + 100n}_{\text{cannot contribute to last two digits}} &+ n^2\\
        {(50 - n)}^2 &= \underbrace{2500 - 100n}_{\text{cannot contribute to last two digits}} &+ n^2\\
    \end{align*}

    Thus in all three cases, the only value that can contribute to values in the last two digits is $n^2$.

    The last equation shows that if we wish to know the last two digits of $n^2$, we may look at ${(50 - n)}^2$ for $0 \leq n \leq 25$ because for $n = 25$ ${(50 - n)}^2 = n^2$. This combined with the fact that $x^2 \pmod{100} \Longleftrightarrow {(x \mod 100)}^2 \pmod{100}$ and $x^2 \pmod{10} \Longleftrightarrow {(x \mod 10)}^2 \pmod{10}$ allows us to exhaustively prove this:

    \begin{minted}{python}
        [a**2 for a in range(0, 26)]
    \end{minted}

    gives the following:

    \begin{minted}{python}
        [0, 1, 4, 9, 16, 25, 36, 49, 64, 81, 100, 121, 144, 169, 196, 225, 256, 289, 324, 361, 400, 441, 484, 529, 576, 625]
    \end{minted}

    from which we can observe the $\{00, e1, e4, 25, o6, e9\}$ pattern.

    \subsection{} \textit{Explain how the result of part \textup{a} can be used to speed up Fermat's factorization method.}

    The Fermat factorization algorithm relies on the function \mintinline{python}{is_square(n)} function, which could be optimized by first checking if the input matches this pattern before computing an expensive square root. However, I'd warn against attempting to prematurely optimize \mintinline{python}{gmpy2.is_square}\ldots

    \begin{minted}{python}
        def _fermat_factor(num):
            a = gmpy2.isqrt(num)
            b2 = gmpy2.square(a) - num

            while not gmpy2.is_square(b2):
                a += 1
                b2 = gmpy2.square(a) - num
            return int(a + gmpy2.isqrt(b2)), int(a - gmpy2.isqrt(b2))
    \end{minted}
\end{document}
